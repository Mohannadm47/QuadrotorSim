\documentclass[compress]{beamer}
\usepackage{etex}
\usetheme{Orebro}

\newcommand{\lectureTitle}{{\rm \LaTeX{}} Template for Reading Assignment}

%\addtolength{\bodyheight}{-2mm} 
%If you think \bodyheight is too tall, you can adjust it like this.

\definecolor{red}{rgb}{1,0,0}\newcommand{\red}{\color{red}}
\definecolor{darkred}{rgb}{0.60,0.11,0.11}\newcommand{\darkred}{\color{darkred}}
\definecolor{darkgreen}{rgb}{0.4,1.0,0.4}\newcommand{\darkgreen}{\color{darkgreen}}
\definecolor{darkyellow}{rgb}{1.0,0.8,0.0}\newcommand{\darkyellow}{\color{darkyellow}}
\definecolor{darkblueone}{rgb}{0.2,0.2,1.0}\newcommand{\darkblueone}{\color{darkblueone}}
\definecolor{gray}{rgb}{0.8,0.8,0.8}\newcommand{\gray}{\color{gray}}
\definecolor{darkgray}{rgb}{0.6,0.6,0.6}\newcommand{\darkgray}{\color{darkgray}}
\definecolor{darkbluea}{rgb}{0.13,0.13,0.44}\newcommand{\darkbluea}{\color{OruBlue}}
\definecolor{darkblue}{rgb}{0.18,0.23,0.53}\newcommand{\darkblue}{\color{darkblue}}
\definecolor{white}{rgb}{1.0,1.0,1.0}\newcommand{\white}{\color{white}}
\definecolor{black}{rgb}{0.0,0.0,0.0}\newcommand{\black}{\color{black}}
\definecolor{ugreen}{rgb}{0.24,0.62,0.24}\newcommand{\ugreen}{\color{ugreen}}
\definecolor{lgreen}{rgb}{0.88,0.91,0.88}\newcommand{\lgreen}{\color{lgreen}}
\definecolor{ured}{rgb}{0.7,0.2,0.2}\newcommand{\ured}{\color{ured}}
\definecolor{lred}{rgb}{0.85,0.6,0.6}\newcommand{\lred}{\color{lred}}
\definecolor{reallylightblue}{HTML}{D4EAFF}\newcommand{\reallylightblue}{\color{reallylightblue}}
\definecolor{lightgray}{HTML}{A7A7A7}\newcommand{\lightgray}{\color{lightgray}}
\definecolor{orange}{HTML}{FF9700}\newcommand{\orange}{\color{orange}}
\definecolor{mygray}{RGB}{185,220,255}\newcommand{\mygray}{\color{mygray}}

\newcommand{\at}{\makeatletter @\makeatother}

\newcommand{\mymovie}[3]{
\begin{center}
\movie[
           width=#1\textwidth,
           autostart
%           poster,
%           loop
           ]
           {
             \centering
             \hspace{-0.1cm}\includegraphics[width=#1\textwidth]{#2}
           }
          {#3}
\end{center}
}


\newcommand{\mymovienocenter}[3]{
\movie[
           width=#1\textwidth,
           autostart
%           poster,
%           loop
           ]
           {
             \centering
             \hspace{-0.1cm}\includegraphics[width=#1\textwidth]{#2}
           }
          {#3}
}

\newcommand{\OruBlue}{\color{OruBlue}}

\newcommand{\changespace}[1]{\renewcommand{\baselinestretch}{#1}\normalsize}

\usepackage{savesym}
\usepackage{xcolor,colortbl}
\usepackage{multimedia}
%\usepackage{CJKutf8}
\usepackage[english]{babel}
%\usepackage{eurosans}
%\usepackage{eurosym}
\usepackage{alltt}
\usepackage{calc}
\usepackage{marvosym}
\usepackage[ruled,vlined]{algorithm2e}
\usepackage{pstricks}
\usepackage[thicklines]{cancel}
\usepackage{empheq}
\usepackage[absolute,overlay]{textpos}

\savesymbol{Cross}
\usepackage{bbding}
\restoresymbol{TXF}{Cross}

\usepackage{tikz}
% \usepackage[absolute,overlay]{textpos}
% \textblockorigin{0mm}{0mm}

\newcommand{\transboxnocenter}[2]{
  \begin{center}
    \fbox{
      \begin{minipage}{#1\textwidth}\vspace{0.1cm}
        #2 \vspace{0.1cm}
      \end{minipage}
    }
  \end{center}
}

\newcommand{\transbox}[2]{
  \begin{center}
    \fbox{
      \begin{minipage}{#1\textwidth}\vspace{0.1cm}
        \centering #2 \vspace{0.1cm}
      \end{minipage}
    }
  \end{center}
}

\newcommand{\nontransbox}[2]{
  \begin{center}
    \fcolorbox{black}{white}{
      \begin{minipage}{#1\textwidth}\vspace{0.1cm}
        \centering #2 \vspace{0.1cm}
      \end{minipage}
    }
  \end{center}
}


\DeclareMathOperator*{\argmin}{arg\,min}
\DeclareMathOperator*{\argmax}{arg\,max}
\newcommand{\mbm}[1]{{\bf #1}}
\newcommand{\dba}[1]{{\OruBlue{\bf #1}}}
\newcommand{\dbb}[1]{{\OruBlue{\bf{\em #1}}}}
\newcommand{\mybox}[2]{
%
	\begin{center}
	%\vspace{-0.2cm}
	\begin{tabular}{|c|}
	\hline
	\begin{minipage}{#1\textwidth}\vspace{0.3cm}
	\centering
	\emph{
	#2
	}
	\vspace{0.3cm}\end{minipage}\\
	\hline
	\end{tabular}
	%\vspace{-0.2cm}
	\end{center}
%
}

\newcommand{\graybox}[1]{
\fboxrule=0pt
\begin{center}
\vspace{0.3cm}
\noindent
\fcolorbox{black}{mygray}{\parbox{0.94\columnwidth}{\vspace{0.3cm}
\begin{minipage}{0.92\columnwidth}\centering
#1
\end{minipage}
\vspace{0.3cm}
}}
\end{center}
\vspace{0.3cm}
}

% gray box
\newcommand{\gbox}[1]{
  \begin{center}
    \fcolorbox{black}{gray}{
      \begin{minipage}[b]{0.98\textwidth}
        \begin{center}
          %\vspace{2mm}
          \begin{minipage}{0.97\textwidth}
            #1 
          \end{minipage}
          \vspace{2mm}
        \end{center}
      \end{minipage}
    }
  \end{center}
}

\newcommand{\wbox}[1]{
  \begin{center}
    \fcolorbox{black}{white}{
      \begin{minipage}[b]{0.98\textwidth}
        \begin{center}
          %\vspace{2mm}
          \begin{minipage}{0.97\textwidth}
            #1 
          \end{minipage}
          \vspace{2mm}
        \end{center}
      \end{minipage}
    }
  \end{center}
}

\mode<presentation>
{
%\useoutertheme{infolines}
\setbeamertemplate{footline}
{
  \hbox{%
  \begin{beamercolorbox}[wd=.9\paperwidth,ht=2.25ex,dp=1ex,center]{title in foot}%
  \usebeamerfont{title in foot}{\OruBlue\copyright{} 2016 \insertshortauthor{} / \"Orebro University -- \url{aass.oru.se}} \end{beamercolorbox}%
\begin{beamercolorbox}[wd=.1\paperwidth,ht=2.25ex,dp=1ex,right]{date in head/foot}%
    \usebeamerfont{date in head/foot}{\OruBlue\insertframenumber{} / \inserttotalframenumber\hspace*{2ex}}
  \end{beamercolorbox}}%
  \vskip0pt%
}
}

% ===========================================================================
\title{Planning \& Scheduling, DT4047}
\subtitle{\lectureTitle{}}
\author[F.~Pecora]{Author Names\\\url{my@email.com, your@email.com}\\\vspace{0.2cm}School of Science and Technology\\\vspace{0.1cm}\"Orebro University, Sweden}

\date{\centering
\vspace{1cm}
\includegraphics[height=45pt]{themeFig/Logo_txt_runt_farg-big.png}
}


%Optional extra graphics on the title page
\titlegraphic{%
  \colorbox{reallylightblue}{\parbox[t][\paperheight][c]{\linewidth}
    {
        %\vspace{1.3cm}
        \begin{center}
        \includegraphics[width=\linewidth]{fig/book.png}
        \end{center}
      %WITH FUNDER LOGO(S)
      %WITH CONTRIBUTORS
      % \vspace{0.3cm}
      % {\OruBlue $^\star$}
      % \begin{minipage}{0.9\linewidth}\changespace{0.6}
      % \scalebox{.6}{{\OruBlue{\em Lecture in partial achievement}}}
      % \scalebox{.6}{{\OruBlue{\em of Associate Professorship}}}
      % %$\;$ \includegraphics[width=0.6\linewidth]{themeFig/kk_eng_colour_tif.png}
      % \end{minipage}
    }%
  }%
}

% %Optional extra graphics on the title page
% \titlegraphic{%
% %  \colorbox{OruBlue}{\parbox[t][\paperheight][c]{\linewidth}
%   \colorbox{reallylightblue}{\parbox[t][\paperheight][c]{\linewidth}
%     {
%       {\flushright
%         \includegraphics[width=\linewidth]{newfig/tl-fragment.png}\\
%       }
%       \vspace{0.3cm}
%       {\black \scalebox{.6}{$^\star$ {\em Contributors:}}}\\\vspace{-0.15cm}
%       {\black $\;\;$ \scalebox{.6}{H.~Andreasson, M.~Cirillo, T.~Stoyanov}}%\\\vspace{-0.15cm}
%     }%
%   }%
% }


% ===========================================================================
\begin{document}
\maketitle

%%%%%%%%%%%%%%%%%%
%% Course intro %%
%%%%%%%%%%%%%%%%%%

\section{Reading Assignment}


% \frame{\frametitle{Course Literature: Books}

% \begin{tabular}{rl}
% \begin{minipage}{0.08\textwidth}\includegraphics[width=\textwidth]{newfig/gnt-cover.jpg}\end{minipage} & \begin{minipage}{0.85\textwidth}{\footnotesize M.~Ghallab, D.~Nau and P.~Traverso. ``Automated Planning: Theory and Practice'', Elsevier, 2004}\end{minipage}\\
% $\;\;$ & \\
% \begin{minipage}{0.08\textwidth}\includegraphics[width=\textwidth]{newfig/slv-cover.jpg}\end{minipage} & \begin{minipage}{0.85\textwidth}{\footnotesize S.~LaValle. ``Planning Algorithms'', Cambridge university press, 2006}\end{minipage}\\
% $\;\;$ & \\
% \begin{minipage}{0.08\textwidth}\includegraphics[width=\textwidth]{newfig/rn-cover.jpg}\end{minipage} & \begin{minipage}{0.85\textwidth}{\darkgray{\footnotesize S.J.~Russell, P.~Norvig. ``Artificial Intelligence: a Modern Approach'' (third edition), Pearson Publishing, 2010}}\end{minipage}\\
% $\;\;$ & \\
% \begin{minipage}{0.08\textwidth}\includegraphics[width=\textwidth]{newfig/cp-cover.jpg}\end{minipage} & \begin{minipage}{0.85\textwidth}{\darkgray{\footnotesize R. Dechter. ``Constraint Processing'', Morgan Kaufmann, 2003}}\end{minipage}
% \end{tabular}
% }


\frame{\frametitle{Reading Assignment --- Action Planning}

Action planning
\vspace{0.5cm}
%\nocite{Korf:2010:AIS:1882723.1882745,gent-walsh-99-tech-rep,chaff-01,ks-96}
%\nocite{blum-furst-gp,ks-96,bb-99,geffner-bonet-hsp,cestaOddiSmithJOH02,lavalle1998r,Likhachev:2009:PLD:1577179.1577184,Karaman.Frazzoli:IJRR11,pecora2012mission,fikes-nilsson-71-STRIPS,Hoffmann:2001:FPS:1622394.1622404}

{\fontsize{6}{2}\selectfont

\begin{thebibliography}{}

\bibitem[Hoffmann and Nebel, 2001]{Hoffmann:2001:FPS:1622394.1622404}
Hoffmann, J. and Nebel, B. (2001).
\newblock The FF planning system: Fast plan generation through heuristic
  search.
\newblock {\em J. Artif. Int. Res.}, 14(1):253--302.

\bibitem[Fikes and Nilsson, 1971]{fikes-nilsson-71-STRIPS}
Fikes, R.~E. and Nilsson, N.~J. (1971).
\newblock {STRIPS: A new approach to theorem proving in problem solving}.
\newblock {\em {Artificial Intelligence}}, 2:189--208.

\bibitem[Blum and Furst, 1997]{blum-furst-gp}
Blum, A. and Furst, M. (1997).
\newblock {Fast Planning Through Planning Graph Analysis}.
\newblock {\em Artificial Intelligence}, pages 281--300.

\bibitem[Bonet and Geffner, 2001]{geffner-bonet-hsp}
Bonet, B. and Geffner, H. (2001).
\newblock {Planning as heuristic search}.
\newblock {\em Artificial Intelligence}, 129(1--2):5--33.

% \bibitem[Kautz et~al., 1996]{ks-96}
% Kautz, H., McAllester, D., and Selman, B. (1996).
% \newblock {Encoding Plans in Propositional Logic}.
% \newblock In {\em {Proceedings of the Fifth International Conference on
%   Principles of Knowledge Representation and Reasoning (KR-96)}}.

\bibitem[Kautz and Selman, 1999]{bb-99}
Kautz, H. and Selman, B. (1999).
\newblock {Unifying SAT-based and Graph-based Planning}.
\newblock In {\em Proceedings of IJCAI-99}.

\end{thebibliography}
}

}


% \frame{\frametitle{Course Literature: Scientific Articles}

% Motion planning
% \vspace{0.5cm}

% {\fontsize{6}{2}\selectfont

% \begin{thebibliography}{}

% \bibitem[Karaman and Frazzoli, 2011]{Karaman.Frazzoli:IJRR11}
% Karaman, S. and Frazzoli, E. (2011).
% \newblock Sampling-based algorithms for optimal motion planning.
% \newblock {\em International Journal of Robotics Research}, 30(7):846--894.

% \bibitem[LaValle, 1998]{lavalle1998r}
% LaValle, S.~M. (1998).
% \newblock {Rapidly-exploring random trees: A new tool for path planning}.
% \newblock Technical report, Computer Science Dept., Iowa State University.

% \bibitem[Likhachev and Ferguson, 2009]{Likhachev:2009:PLD:1577179.1577184}
% Likhachev, M. and Ferguson, D. (2009).
% \newblock Planning long dynamically feasible maneuvers for autonomous vehicles.
% \newblock {\em Int. J. Rob. Res.}, 28(8):933--945.

% \end{thebibliography}
% }

% }

% \addtocounter{framenumber}{-1}
% \frame{\frametitle{Course Literature: Scientific Articles}

% Scheduling and its integration with planning
% \vspace{0.5cm}

% {\fontsize{6}{2}\selectfont

% \begin{thebibliography}{}

% \bibitem[Cesta et~al., 2002]{cestaOddiSmithJOH02}
% Cesta, A., Oddi, A., and Smith, S.~F. (2002).
% \newblock A constraint-based method for project scheduling with time windows.
% \newblock {\em Journal of Heuristics}, 8(1):109--136.

% \bibitem[Pecora et~al., 2012]{pecora2012mission}
% Pecora, F., Cirillo, M., and Dimitrov, D. (2012).
% \newblock On mission-dependent coordination of multiple vehicles under spatial
%   and temporal constraints.
% \newblock In {\em IEEE/RSJ International Conference on Intelligent Robots and
%   Systems (IROS)}.

% \end{thebibliography}
% }

% }

\frame{
\frametitle{Your Task}
\begin{itemize}
\item Read a paper, prepare a \dba{15 min presentation}
\begin{itemize}
\item present it at lecture on Wed Feb 24 \at 10:15--12:00 
\item groups possible, maximum \dba{two people}
\end{itemize}
\item Select a paper from the list
\begin{itemize}
\item $[$Fikes and Nilsson, 1971$]$\\{\em Seminal paper on AI Planning (STRIPS)}\vspace{0.1cm}
\item $[$Blum and Furst, 1997$]$\\{\em Seminal paper on Graphplan}\vspace{0.1cm}
\item $[$Hoffmann and Nebel, 2001$]$\\{\em Using planning graphs in heuristic search}\vspace{0.1cm}
\item $[$Bonet and Geffner, 2001$]$\\{\em Seminal paper on planning as heuristic search}\vspace{0.1cm}
\item $[$Kautz and Selman, 1999$]$\\{\em Using SAT solvers for search in planning graphs}
\end{itemize}
\end{itemize}
}

\frame{
\frametitle{Contents of Presentation}
\begin{itemize}
\item \dba{Motivation:} how is the work motivated and what high-level problem does it solve?
\item \dba{Problem formulation:} state the problem solved by the approach in formal terms
\item \dba{Key techniques:} state the essential features of the proposed algorithm --- focus on one or two key technical aspects of the proposed method
\item \dba{Evaluation:} how is the proposed method evaluated? What metrics are used and why?
\item \dba{Future work:} what open questions does the proposed technique/solution/method lead to?
\item \dba{Paper quality:} comment on the quality of the paper --- is it well written, clearly organized, convincing, fun to read, \dots{}?
\end{itemize}
}


\end{document}


