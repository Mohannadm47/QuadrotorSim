\documentclass[12pt,fleqn]{article}

\newcommand{\mbm}[1]{\mbox{\boldmath $#1$}}
\newcommand{\norm}[1]{\lVert#1\rVert}


\usepackage[small,hang]{caption2}
\usepackage{graphicx}
\usepackage{amsmath}
\usepackage{amssymb}
\usepackage{hyperref}
\usepackage{pdfpages}
\usepackage[          % set page and margin sizes
  a4paper,
  top=7mm,
  bottom=7mm,
  inner=20mm,
  outer=20mm,
  bindingoffset=0mm,
  head=10mm,
  foot=10mm,
  headsep=15mm,
  footskip=15mm,
  includeheadfoot,
]{geometry}
\usepackage{fancyhdr}
\pagestyle{fancy}
\fancyhead{}
\fancyfoot{}
\fancyhead[LO,LE]{Project Work}
\fancyhead[CO,CE]{--- Quadcopter Stabilization and Tracking ---}
\fancyhead[RO,RE]{May $27$, $2016$}
\fancyfoot[RO, LE] {\thepage}


%%%%%%%%%%%%%
%%Titlepage%%
%%%%%%%%%%%%%%

\begin{document}
\thispagestyle{empty}
\noindent\makebox[\linewidth]{\rule{.75\paperwidth}{0.4pt}}
\begin{center}
{\Large \textbf{{\"O}rebro University}}
\end{center}
\vspace{-4mm}
\noindent\makebox[\linewidth]{\rule{.75\paperwidth}{0.4pt}} \\
\vspace{4cm}
\begin{center}
{\Large\sffamily\bfseries
Project Work: \\[1ex]
Quadcopter Stabilization and Tracking \\[2ex]
}

\vspace{6cm}
\noindent\makebox[\linewidth]{\rule{.75\paperwidth}{0.4pt}}
{\large
\renewcommand{\arraystretch}{1.5}
  \begin{tabular}{l}
   \textbf{Anders Wikstr\"{o}m}\\
   (anders.wikstrom.88@gmail.com)\\
   \textbf{Chittaranjan Srinivas Swaminathan}\\
   (chitt@live.in)\\
 \end{tabular}}
\end{center}
\vspace{-2mm}
\noindent\makebox[\linewidth]{\rule{.75\paperwidth}{0.4pt}}
\newpage

\tableofcontents

\newpage


%%%%%%%%%%%%%%%%%%%%%%%%%
%%Beginning of document%%
%%%%%%%%%%%%%%%%%%%%%%%%%

\section*{Abstract}
A Quadcopter are non-holonomic and non-linear system that is highly
difficult to control. The aim of this project is to understand the
model and design a inverse dynamics controller for altitude stability
and trajectory tracking. In this report we present a cascaded
PID-Inverse Dynamics controller. With this the quadcoptor is able to
track trajectories with an error less than a millimeter.

\section{Introduction}

\section{Background}
In this section we describe the model of the quadcopter used in the
simulation and control. The following subsections detail the different
components of the dynamics.

\subsection{Forces and Torques from the motors}

\subsection{Equations of motion}

The linear dynamics of the quadrotor system can be described by the
following equation:

$$ m \mbm{\ddot{x}} = \mbm{G} + \mbm{^BR_IT_{B}} + \mbm{F_{drag}} $$

where, 

$$ G = \begin{bmatrix} 0 \\ 0 \\ -mg \end{bmatrix} $$

The rotational dynamics are given by Euler's equations for rigid body
dynamics. It

$$ \mbm{I\dot{\omega}} + \mbm{\omega} \times (\mbm{I\omega}) $$



\section{Approach}

\section{Results}

\section{Conclusion}

  \textbf{Answer}: $$ \mbm{M} = \begin{bmatrix}1&&0&&0&&0 \\ 0&&0.5&&0&&0 \\ 0&&0&&1&&0 \\ 0&&0&&0&&1 \end{bmatrix} $$ \\ such that, $\mbm{MA}$ gives the required result.   

\end{document}